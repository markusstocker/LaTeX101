\documentclass{beamer}
\usetheme{Boadilla}
\usecolortheme{sidebartab}

\usepackage{hyperref}
\usepackage{showexpl} 

\lstloadlanguages{[LaTeX]Tex} 
\lstset{% 
     basicstyle=\ttfamily\small, 
     commentstyle=\itshape\ttfamily\small, 
     showspaces=false, 
     showstringspaces=false, 
     breaklines=true, 
     breakautoindent=true, 
     captionpos=t,
     pos=r
} 

\title{Introduction to writing with LaTeX}
\author{Markus Stocker}
\date{May 12, 2017}

\begin{document}
\maketitle

\begin{frame}
  \frametitle{Outline}
  
  \begin{itemize}
  \item Morning lecture
  \begin{itemize}
  \item What is \LaTeX
  \item Some \LaTeX~history
  \item Motivating \LaTeX 
  \item \LaTeX~software environment
  \item Elements of \LaTeX~documents
  \item Scientific documents with journal \LaTeX~templates
  \item \LaTeX~for slides and posters
  \item Collaborative writing and versioning of \LaTeX~documents
  \end{itemize}
  \item Afternoon hands-on
  \begin{itemize}
  \item Develop your \LaTeX~manuscript
  \item Style your manuscript with journal templates
  \item Revise your manuscript with track changes
  \item Create slides and a poster to present your work
  \item Collaborative writing with your co-authors
  \end{itemize}
  \end{itemize}

\end{frame}

\begin{frame}
  \frametitle{Schedule}
  
  \begin{center}
  \begin{tabular}{ll}
  10.00 - 11.30 & Lecture \\
  11.30 - 12.30 & \emph{Lunch} \\
  12.30 - 13.15 & Hands-on I \\
  13.15 - 13.30 & \emph{Break} \\
  13.30 - 14.15 & Hands-on II \\
  14.15 - 14.45 & \emph{Coffee break} \\
  14.45 - 15.30 & Hands-on III \\
  15.30 - 15.45 & \emph{Break} \\
  15.45 - 16.30 & Hands-on IV \\
  16.30 - 17.00 & Closing
  \end{tabular}
  \end{center}

\end{frame}

\begin{frame}
  \frametitle{About me}
  
  \begin{itemize}
  \item Postdoc with PANGAEA at MARUM
  \item PhD in environmental informatics at University of Eastern Finland
  \item MSc in informatics at University of Zurich
  \item MSc in environmental science at University of Eastern Finland \emph{(soon)}
  \item My history with \LaTeX~goes back to 2001 when ...
  \end{itemize}

\end{frame}

\begin{frame}
  \frametitle{What is \LaTeX}
  
  \begin{itemize}
  \item Document preparation system
  \item Most often used for technical or scientific documents
  \item Worry less about style and more about content
  \item Write plain text rather than formatted text
  \item Leave document design to designers
  \item Free software
  \item Available for Windows, Mac OS, Linux, Online
  \end{itemize}
  
  \begin{flushright}
  \url{https://www.latex-project.org}
  \end{flushright}

\end{frame}


\begin{frame}[fragile]
  \frametitle{What is \LaTeX}
	
  \begin{itemize}
  \item \textbf{Markup tagging} is central to writing with \LaTeX
  \item Label parts of the document using tags, e.g. \lstinline!\textit{}!
  \item It is used to do things like
  \begin{itemize}
  \item Define document structure, e.g. chapters, sections
  \item Style text, e.g. italic, tables
  \item Cite, footnote, cross-reference, ...
  \end{itemize}
  \item Anyone familiar with HTML?
  \end{itemize}
	
\end{frame}

\frame[containsverbatim]{ 
  \frametitle{Markup tagging}
		
  \begin{LTXexample}
  \textit{Example} 
  markup 
  \underline{tagging}
  \end{LTXexample}
}

\frame[containsverbatim]{ 
  \frametitle{Markup tagging}
	
  \begin{LTXexample}
  \begin{itemize}
  \item Eggs
  \item Milk
  \item Cheese
  \item Carrots
  \end{itemize}
  \end{LTXexample}
}

\begin{frame}
  \frametitle{A bit of history}
  
  \begin{itemize}
  \item \LaTeX~was authored by L. Lamport, an American computer scientist
  \item Originally released in 1985
  \end{itemize}

\end{frame}

\end{document}